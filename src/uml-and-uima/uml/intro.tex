
\chapter{UML Design}

In this task, you are required to answer some questions about UML. Once you've
answered all the questions, remembered to convert whatever format your document
is to PDF/DOC/DOCX, and name it as hw1-ID-uml.pdf, hw1-ID-uml.doc,
hw1-ID-uml.docx, where ID is your Andrew ID, and put it under
\texttt{src/main/resources/docs}. Moreover, don't forget to put your name and
Andrew ID at the beginning of the document.

\section{Domain diagram for IntelligentInformationSystem}

``An IntelligentInformationSystem is composed of a sequence of data processing
operations or phases. Each phase accepts certain data types as input and
produces certain data types as output. Each phase can be implemented by any
number of algorithms or options. Each option is implemented by a specific Java
class. Each option may have any number of configuration parameters; each
configuration parameter has some set of acceptable values.''

Draw a UML Domain Diagram to represent the domain concepts, associations (with
multiplicities) and attributes expressed in the description above.

\section{Domain diagram for AnalysisEngine}

``An AnalysisEngine is composed of a sequence of algorithms or options. Each
option accepts certain data types as input and produces certain data types as
output. Each option is implemented by a specific Java class. Each option has
some number of configuration parameters; each configuration parameter has a
specific assigned value.''

Draw a UML Domain Diagram to represent the domain concepts, associations (with
multiplicities) and attributes expressed in the description above.

\section{Sequence Diagram}

There is a one-to-many relationship between IntelligentInformationSystem and
AnalysisEngine. Assume that an IntelligentInformationSystem has the
responsibility to produce a set of AnalysisEngines that represent all of the
possible data flows in the IntelligentInformationSystem. Design a method with
this signature:

\small
\begin{verbatim}
ArrayList<CollectionProcessingEngine> instantiateEngines (
                          IntelligentInformationSystem iis);
\end{verbatim}
\normalsize

Draw a UML Sequence Diagram to show the sequence of messages required to a) read
the information from the IntelligentInformationSystem instance, b) instantiate
the corresponding AnalysisEngine instances, and c) store the AnalysisEngine
instances in a List, which is the final output of the program. The message and
return value for this use case are illustrated below.

\begin{figure}[h]
\centering
\begin{pspicture}(0,-2.315)(6.1384373,2.335)
\usefont{T1}{ptm}{m}{n}
\rput(3.6746874,1.81){\psframebox[linewidth=0.04,framesep=0.3]{IntelligentInformationSystem}}
\usefont{T1}{ptm}{m}{n}
\rput(1.6684375,0.61){instantiateEngines}
\usefont{T1}{ptm}{m}{n}
\rput(1.7542187,-0.99){List<AnalysisEngine>}
\psframe[linewidth=0.04,dimen=outer](4.0,0.505)(3.6,-1.495)
\psline[linewidth=0.04cm,linestyle=dashed,dash=0.16cm 0.16cm](3.8,1.305)(3.8,0.505)
\psline[linewidth=0.04cm,linestyle=dashed,dash=0.16cm 0.16cm](3.8,-1.495)(3.8,-2.295)
\psline[linewidth=0.04cm,arrowsize=0.05291667cm 5.0,arrowlength=1.4,arrowinset=0.4]{->}(0.0,0.305)(3.6,0.305)
\psline[linewidth=0.04cm,linestyle=dashed,dash=0.16cm 0.16cm,arrowsize=0.05291667cm 5.0,arrowlength=1.4,arrowinset=0.4]{->}(3.6,-1.495)(0.0,-1.495)
\end{pspicture} 
\caption{The message and return value for this use case}
\end{figure}

