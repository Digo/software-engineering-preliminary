% 
% This work is licensed under the Creative Commons Attribution-ShareAlike 3.0
% Unported License. To view a copy of this license, visit
% http://creativecommons.org/licenses/by-sa/3.0/.
%
% author: Zi Yang <ziy@cs.cmu.edu>
% date: 09/02/2012
%
\documentclass[oneside]{memoir}

\renewcommand{\chaptername}{Task}

\usepackage{times}

\usepackage{titlesec}
\titleformat{\section}{\normalfont\Large\bfseries}{Task \thesection}{1em}{}

\usepackage{url}
\usepackage{hyperref}

\usepackage{graphicx}
\graphicspath{{../../fig/uml-and-uima/}}

\usepackage{pstricks}
\usepackage{epsfig}

\usepackage{listings}
\usepackage{color}
\definecolor{dkgreen}{rgb}{0,0.6,0}
\definecolor{gray}{rgb}{0.5,0.5,0.5}
\definecolor{lightblue}{rgb}{0.95,0.95,1}
\definecolor{mauve}{rgb}{0.58,0,0.82}
\lstset{
  basicstyle=\ttfamily\small,
  numbers=left,
  numberstyle=\tiny\color{gray},
  stepnumber=2, 
  numbersep=5pt,
  backgroundcolor=\color{lightblue},
  showspaces=false,
  showstringspaces=false,
  showtabs=false,
  frame=lines,
  rulecolor=\color{black},
  tabsize=2,
  captionpos=b,
  breaklines=true,
  breakatwhitespace=false,
  title=\lstname,
  keywordstyle=\color{blue},
  commentstyle=\color{dkgreen},
  stringstyle=\color{mauve},
  escapeinside={\%*}{*)},
  morekeywords={*,...},
}
\usepackage{letltxmacro}
\makeatletter
\LetLtxMacro\@@lst@inputlisting\lst@inputlisting
\renewcommand\lst@inputlisting[2][]{%
  \try@listingspath{#2}%
  \if@tempswa
    \typeout{Using \@foundlisting}%
    \@@lst@inputlisting[#1]{\@foundlisting}%
  \else
    \typeout{Missing file #2}\endgroup
  \fi}
\def\listingspath#1{\def\@listingspath{#1}}
\listingspath{}
\def\try@listingspath#1{%
  \@tempswafalse
  \expandafter\@tfor\expandafter\next
  \expandafter:\expandafter=\@listingspath\do
  {\if@tempswa\@break@tfor\fi
   \IfFileExists{\next/#1}{\@tempswatrue\xdef\@foundlisting{\next/#1}}{}}%
}
\makeatletter
\listingspath{{../../lst/uml-and-uima/}}

\definecolor{shadecolor}{gray}{0.9}

\newenvironment{qa}
{\begin{shaded}\begin{itemize}}
{\end{itemize}\end{shaded}}

\title{{\bfseries 11-791 Design and Engineering of Intelligent Information
System Fall 2012\\Homework 1}\\
\vspace{1em}
\itshape\rmfamily UML Design and Named Entity Recognition Implementation with
UIMA SDK}

\date{}

\begin{document}

\begin{titlingpage}
\maketitle

\hspace{-0.1\textwidth}
\begin{minipage}{1.2\textwidth}
\vspace{-5em}
\textbf{Important dates}
\begin{itemize}

\item \textbf{Hand out: October 2.}\footnote{This version was built on September
28, 2012}

\item \textbf{Turn in: October 16.} Besides all your Java source codes, and UIMA
type system, collection reader and analysis engine descriptors, which are
essential to your UIMA collection processing engine, you are also required to
answer UML design related questions, write a report for your implementation of
the named entity recognizer, and include the Javadocs.

If you have ever looked into your \verb|target/| directory or Maven
course-release repository for Homework 0, then you will find that Maven will
package the binary files, source files, and Javadocs into different jars, and
deploy them on the server, which means all you need to do is to put all the
source codes and descriptors, as well as the documents, in the right place of
your hw1-ID project. You should organize your project in the same hierarchy as
shown below\footnote{To simplify your submission process and our evaluation
process, we ask you to create \texttt{src/main/resources/docs} for your
documents. But you should keep in mind it is NOT a good practise to have
documentation within a \texttt{src} folder.}:

\small
\begin{verbatim}
hw1-ID
|- pom.xml
`- src
     `- main
        |- java
      |   `- **/*.java            /* your Java codes go into this folder 
      |                           * as you did before */
        `- resources
         |- CpeDescriptor.xml    /* the entry point for your cpe */
         |- **/*.xml             /* all your descriptors go into the 
         |                        * resources folder */
         `- docs
            |- hw1-ID-uml.pdf    /* your UML design writeup */
            `- hw1-ID-report.pdf /* your report for the pipeline */

\end{verbatim}
\normalsize

\end{itemize}

\end{minipage}
\hspace{-0.1\textwidth}

\hspace{-0.1\textwidth}
\begin{minipage}{1.2\textwidth}

Several notes about organizing your Maven project and other additional
information:

\begin{enumerate}

\item \textbf{Submission:} The same way as you did for Homework 0 (set up GitHub
repo, create Maven project, write your code, submit to Maven repo), except that
the name has changed to hw1-ID.

\item \textbf{Your source files and descriptors:} \verb|**/*.java| and
\verb|**/*.xml| tell you that you don't need to flatten your folder hierarchy,
instead we encourage you to place your Java codes in the right package, and
similarly, you can create subfolders for different types of descriptors, e.g.,
\verb|src\main\resources\descriptors\ner| for all the analysis engine
descriptors related to named entity recognition task. We will look into your jar
packages.

One very special descriptor is the \verb|CpeDescriptor.xml|, which is the entry
point for the entire pipeline. You are required to name your CPE descriptor
exactly the same way, and place it in the right place for us to evaluate your
code.

\item \textbf{UML design answers and report:} We will pull out your documents
from your jar files, and Prof. Eric Nyberg will look into your report and
answers for the questions, so that remember to include your ID as part of file
names, and put your name and ID in your document. You can submit either PDF
files or DOC/DOCX files (hw1-ID-uml.doc or .docx in the latter case).

\item \textbf{Javadocs:} Please refer to any best practise (yes, there might be
more than one) to write your Javadoc, e.g.,
\url{http://www.oracle.com/technetwork/java/javase/documentation/index-137868.html}.
We expect you to describe your major components at the class level of your
Javadoc, and put additional comments on the most important methods if any. It
might be a good idea to include your UML design in your Javadoc if you think the
developers can have a general idea by just looking at your diagram.

\item \textbf{Performance evaluation:} We believe you may have different ideas
about how to design the type system, and how to implement a name entity
recognizer, which means we won't prepare a shared type system for you to build
components on (but we will have one in Homework 2). Therefore, you are required
to write a collection reader to load the input text and a CAS consumer to
generate your output (both are independent of your specific type system), and
our evaluation will based on the output from your CAS consumer, similar as we
did for Homework 0.

\end{enumerate}

\textbf{Useful information}
\begin{enumerate}
\item We encourage you to start Homework 1 as early as possible, especially if
you haven't gotten a chance to browse the UIMA portal.

\item For any question regarding the UML design part of Homework 1, please send
mails directly to Prof. Eric Nyberg
(\href{mailto:ehn@cs.cmu.edu}{\nolinkurl{ehn@cs.cmu.edu}}), and for other
questions, please send us mails: Zi Yang
(\href{mailto:ziy@cs.cmu.edu}{\nolinkurl{ziy@cs.cmu.edu}}) or Rui Liu
(\href{mailto:ruil@cs.cmu.edu}{\nolinkurl{ruil@cs.cmu.edu}}).

\item Again, both source files and pdf file of this assignment are
publicly available on my GitHub

\url{http://github.com/ziy/software-engineering-preliminary}

Please feel free to fork the project and send a pull request back to me as some
of you did for Homework 0 for any error. Or you can just report an issue at

\url{http://github.com/ziy/software-engineering-preliminary/issues}

\end{enumerate}

\end{minipage}
\hspace{-0.1\textwidth}

\end{titlingpage}


\chapter{UML Design}

In this task, you are required to answer some questions about UML. Once you've
answered all the questions, remembered to convert whatever format your document
is to PDF/DOC/DOCX, and name it as hw1-ID-report.pdf, hw1-ID-report.doc,
hw1-ID-report.docx, where ID is your Andrew ID, and put it under
\texttt{src/main/resources/docs}. Moreover, don't forget to put your name and
Andrew ID at the beginning of the document.

\section{Domain diagram for IntelligentInformationSystem}

``An IntelligentInformationSystem is composed of a sequence of data processing
operations or phases. Each phase accepts certain data types as input and
produces certain data types as output. Each phase can be implemented by any
number of algorithms or options. Each option is implemented by a specific Java
class. Each option may have any number of configuration parameters; each
configuration parameter has some set of acceptable values.''

Draw a UML Domain Diagram to represent the domain concepts, associations (with
multiplicities) and attributes expressed in the description above.

\section{Domain diagram for AnalysisEngine}

``An AnalysisEngine is composed of a sequence of algorithms or options. Each
option accepts certain data types as input and produces certain data types as
output. Each option is implemented by a specific Java class. Each option has
some number of configuration parameters; each configuration parameter has a
specific assigned value.''

Draw a UML Domain Diagram to represent the domain concepts, associations (with
multiplicities) and attributes expressed in the description above.

\section{Sequence Diagram}

There is a one-to-many relationship between IntelligentInformationSystem and
AnalysisEngine. Assume that an IntelligentInformationSystem has the
responsibility to produce a set of AnalysisEngines that represent all of the
possible data flows in the IntelligentInformationSystem. Design a method with
this signature:

\small
\begin{verbatim}
ArrayList<CollectionProcessingEngine> instantiateEngines (
                          IntelligentInformationSystem iis);
\end{verbatim}
\normalsize

Draw a UML Sequence Diagram to show the sequence of messages required to a) read
the information from the IntelligentInformationSystem instance, b) instantiate
the corresponding AnalysisEngine instances, and c) store the AnalysisEngine
instances in a List, which is the final output of the program. The message and
return value for this use case are illustrated below.

\begin{figure}[h]
\centering
\begin{pspicture}(0,-2.315)(6.1384373,2.335)
\usefont{T1}{ptm}{m}{n}
\rput(3.6746874,1.81){\psframebox[linewidth=0.04,framesep=0.3]{IntelligentInformationSystem}}
\usefont{T1}{ptm}{m}{n}
\rput(1.6684375,0.61){instantiateEngines}
\usefont{T1}{ptm}{m}{n}
\rput(1.7542187,-0.99){List<AnalysisEngine>}
\psframe[linewidth=0.04,dimen=outer](4.0,0.505)(3.6,-1.495)
\psline[linewidth=0.04cm,linestyle=dashed,dash=0.16cm 0.16cm](3.8,1.305)(3.8,0.505)
\psline[linewidth=0.04cm,linestyle=dashed,dash=0.16cm 0.16cm](3.8,-1.495)(3.8,-2.295)
\psline[linewidth=0.04cm,arrowsize=0.05291667cm 5.0,arrowlength=1.4,arrowinset=0.4]{->}(0.0,0.305)(3.6,0.305)
\psline[linewidth=0.04cm,linestyle=dashed,dash=0.16cm 0.16cm,arrowsize=0.05291667cm 5.0,arrowlength=1.4,arrowinset=0.4]{->}(3.6,-1.495)(0.0,-1.495)
\end{pspicture} 
\caption{The message and return value for this use case}
\end{figure}




\chapter{UML Design}

In this task, you are required to answer some questions about UML. Once you've
answered all the questions, remembered to convert whatever format your document
is to PDF/DOC/DOCX, and name it as hw1-ID-report.pdf, hw1-ID-report.doc,
hw1-ID-report.docx, where ID is your Andrew ID, and put it under
\texttt{src/main/resources/docs}. Moreover, don't forget to put your name and
Andrew ID at the beginning of the document.

\section{Domain diagram for IntelligentInformationSystem}

``An IntelligentInformationSystem is composed of a sequence of data processing
operations or phases. Each phase accepts certain data types as input and
produces certain data types as output. Each phase can be implemented by any
number of algorithms or options. Each option is implemented by a specific Java
class. Each option may have any number of configuration parameters; each
configuration parameter has some set of acceptable values.''

Draw a UML Domain Diagram to represent the domain concepts, associations (with
multiplicities) and attributes expressed in the description above.

\section{Domain diagram for AnalysisEngine}

``An AnalysisEngine is composed of a sequence of algorithms or options. Each
option accepts certain data types as input and produces certain data types as
output. Each option is implemented by a specific Java class. Each option has
some number of configuration parameters; each configuration parameter has a
specific assigned value.''

Draw a UML Domain Diagram to represent the domain concepts, associations (with
multiplicities) and attributes expressed in the description above.

\section{Sequence Diagram}

There is a one-to-many relationship between IntelligentInformationSystem and
AnalysisEngine. Assume that an IntelligentInformationSystem has the
responsibility to produce a set of AnalysisEngines that represent all of the
possible data flows in the IntelligentInformationSystem. Design a method with
this signature:

\small
\begin{verbatim}
ArrayList<CollectionProcessingEngine> instantiateEngines (
                          IntelligentInformationSystem iis);
\end{verbatim}
\normalsize

Draw a UML Sequence Diagram to show the sequence of messages required to a) read
the information from the IntelligentInformationSystem instance, b) instantiate
the corresponding AnalysisEngine instances, and c) store the AnalysisEngine
instances in a List, which is the final output of the program. The message and
return value for this use case are illustrated below.

\begin{figure}[h]
\centering
\begin{pspicture}(0,-2.315)(6.1384373,2.335)
\usefont{T1}{ptm}{m}{n}
\rput(3.6746874,1.81){\psframebox[linewidth=0.04,framesep=0.3]{IntelligentInformationSystem}}
\usefont{T1}{ptm}{m}{n}
\rput(1.6684375,0.61){instantiateEngines}
\usefont{T1}{ptm}{m}{n}
\rput(1.7542187,-0.99){List<AnalysisEngine>}
\psframe[linewidth=0.04,dimen=outer](4.0,0.505)(3.6,-1.495)
\psline[linewidth=0.04cm,linestyle=dashed,dash=0.16cm 0.16cm](3.8,1.305)(3.8,0.505)
\psline[linewidth=0.04cm,linestyle=dashed,dash=0.16cm 0.16cm](3.8,-1.495)(3.8,-2.295)
\psline[linewidth=0.04cm,arrowsize=0.05291667cm 5.0,arrowlength=1.4,arrowinset=0.4]{->}(0.0,0.305)(3.6,0.305)
\psline[linewidth=0.04cm,linestyle=dashed,dash=0.16cm 0.16cm,arrowsize=0.05291667cm 5.0,arrowlength=1.4,arrowinset=0.4]{->}(3.6,-1.495)(0.0,-1.495)
\end{pspicture} 
\caption{The message and return value for this use case}
\end{figure}




\chapter{UML Design}

In this task, you are required to answer some questions about UML. Once you've
answered all the questions, remembered to convert whatever format your document
is to PDF/DOC/DOCX, and name it as hw1-ID-report.pdf, hw1-ID-report.doc,
hw1-ID-report.docx, where ID is your Andrew ID, and put it under
\texttt{src/main/resources/docs}. Moreover, don't forget to put your name and
Andrew ID at the beginning of the document.

\section{Domain diagram for IntelligentInformationSystem}

``An IntelligentInformationSystem is composed of a sequence of data processing
operations or phases. Each phase accepts certain data types as input and
produces certain data types as output. Each phase can be implemented by any
number of algorithms or options. Each option is implemented by a specific Java
class. Each option may have any number of configuration parameters; each
configuration parameter has some set of acceptable values.''

Draw a UML Domain Diagram to represent the domain concepts, associations (with
multiplicities) and attributes expressed in the description above.

\section{Domain diagram for AnalysisEngine}

``An AnalysisEngine is composed of a sequence of algorithms or options. Each
option accepts certain data types as input and produces certain data types as
output. Each option is implemented by a specific Java class. Each option has
some number of configuration parameters; each configuration parameter has a
specific assigned value.''

Draw a UML Domain Diagram to represent the domain concepts, associations (with
multiplicities) and attributes expressed in the description above.

\section{Sequence Diagram}

There is a one-to-many relationship between IntelligentInformationSystem and
AnalysisEngine. Assume that an IntelligentInformationSystem has the
responsibility to produce a set of AnalysisEngines that represent all of the
possible data flows in the IntelligentInformationSystem. Design a method with
this signature:

\small
\begin{verbatim}
ArrayList<CollectionProcessingEngine> instantiateEngines (
                          IntelligentInformationSystem iis);
\end{verbatim}
\normalsize

Draw a UML Sequence Diagram to show the sequence of messages required to a) read
the information from the IntelligentInformationSystem instance, b) instantiate
the corresponding AnalysisEngine instances, and c) store the AnalysisEngine
instances in a List, which is the final output of the program. The message and
return value for this use case are illustrated below.

\begin{figure}[h]
\centering
\begin{pspicture}(0,-2.315)(6.1384373,2.335)
\usefont{T1}{ptm}{m}{n}
\rput(3.6746874,1.81){\psframebox[linewidth=0.04,framesep=0.3]{IntelligentInformationSystem}}
\usefont{T1}{ptm}{m}{n}
\rput(1.6684375,0.61){instantiateEngines}
\usefont{T1}{ptm}{m}{n}
\rput(1.7542187,-0.99){List<AnalysisEngine>}
\psframe[linewidth=0.04,dimen=outer](4.0,0.505)(3.6,-1.495)
\psline[linewidth=0.04cm,linestyle=dashed,dash=0.16cm 0.16cm](3.8,1.305)(3.8,0.505)
\psline[linewidth=0.04cm,linestyle=dashed,dash=0.16cm 0.16cm](3.8,-1.495)(3.8,-2.295)
\psline[linewidth=0.04cm,arrowsize=0.05291667cm 5.0,arrowlength=1.4,arrowinset=0.4]{->}(0.0,0.305)(3.6,0.305)
\psline[linewidth=0.04cm,linestyle=dashed,dash=0.16cm 0.16cm,arrowsize=0.05291667cm 5.0,arrowlength=1.4,arrowinset=0.4]{->}(3.6,-1.495)(0.0,-1.495)
\end{pspicture} 
\caption{The message and return value for this use case}
\end{figure}




\newpage

\section*{Appendix: GENETAG Annotation Guidelines}

The following are some rules about which words are considered as part of a single gene/protein mention.

\begin{enumerate}

\item Mutants

\emph{p53 mutant}

\item Parens at start or end when embedded in the name

\emph{(IGG) receptor}

\item Motifs, elements, and domains \textbf{with a gene name}

\emph{POU domain\\
src domains\\
RXR-responsive element\\
Ras-responsive enhancer}

but not

serum response element\\
AC element\\
B-cell-specific enhancer element\\
dioxin responsive transcriptional enhancer

\item Plurals and families

\emph{immunoglobulins}

\item Fusion proteins

\emph{p53 mdm2 fusion}

\item The words light/heavy chain, monomer, codon, region, exon, orf, cdna, reporter gene,
antibody, complex, gene product, mrna, oligomer, chemokine, subunit, peptide, message,
transactivator, homolog, binding site, enhancer, element, allele, isoform, intron,
promoter, operon, etc. \textbf{with a gene name}.

\item Particular qualifiers such as alpha, beta, I, II, etc. \textbf{with a gene name}

For example, \emph{topo} is not an allowable alternative to \emph{topo II}

\item If the context suggests that a word is necessary, require that word in the allowable
alternatives, even if it is still a name without the word.

\emph{rabies immunoglobulin (RIG) (\textbf{not} \emph{immunoglobulin}}\\
designated \emph{HUG1}, for \emph{Hox11 Upstream Gene} (\textbf{not} \emph{Hox11})

\item Viral promoters, LTRs and enhancers \textbf{with specific virus name}

\emph{HIV long terminal repeat\\
Simian virus 40 promoter\\
HPV 18 enhancer}

\item Antigen receptor region gene segment genes

\emph{CH genes\\
Tamarin variable region genes}

\item Peptide hormones

\emph{Vasopressin, prolactin, FSH}

\item Embedded names – tag \textbf{only the gene/protein part of the name}

\emph{p53-mediated}

\end{enumerate}

The following generally \textbf{do not} qualify as gene name mentions:

\begin{enumerate}

\item Generic terms

\emph{zinc finger} alone (but \emph{zinc finger protein} is an accepted gene/protein mention)

\item Mutations

p53 mutations

\item Parens at start and end which `wraps' the whole gene name

(\emph{IGG})

\item TATA boxes

\item Receptors: if a receptor is specified, the gene name without ``receptor'' is not
considered to be a valid alternative.

\item Synonyms: if a synonym is given in the sentence which implies certain words are
necessary to the gene name, they will be required in the alternatives

For \emph{rabies immunoglobin (RIG)}, ``immunoglobin'' alone will not
be a valid alternative because RIG implies that ``rabies'' is part of the name in this
context.

\item Non-peptide hormones

\item DNA and protein sequences

\end{enumerate}


\end{document}
