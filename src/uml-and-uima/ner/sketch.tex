
\section{Sketching your NER pipeline}

We assume you are clear about the task, the input format, and the output, and
now it's time to sketch your collection processing engine. How do you plan to
design the type system? How do you plan to deal with input file and generate
output file? What approach(es) are you going to use to identify the named
entities? Do you also consider to incorporate any additional resources?

In the next task, you will practise to solve the problem within the UIMA
framework.

\begin{qa}

\item[Q1] I still have no idea about how named entities can be extracted from
text? Could you share me some ideas?

\item[A1] Named entity recognition is a long-studied problem in NLP, which means
you can easily find many influential papers to propose novel ideas to tackle
this issue every year, including specific to the biological domain. For example,

\begin{enumerate}
\item You can try to look for lexical/grammatical heuristics for the gene
mentions, and summarize them in a rule-based algorithm. For example, many people
find part-of-speech is very useful information to identity not only the named
entities but also the text spans that cannot be named entities.
\item If you are familiar with machine learning, then you could try to use a
sequence model like HMM or CRF to predict the state of each token, based on the
features you might extract.
\item If you are familiar with biological knowledge resources, esp. gene name
resources, you can definitely write a wrapper for the resource you think are
relevant to the task, and look up candidate named entities via your wrapper.
\item etc.
\end{enumerate}

We encourage you to try some novel ideas for your NER. You can incorporate an
interesting feature in your learning framework, or you can try to combine
multiple approaches in a more principled way. We know that each of you has a
different background. So don't worry if you don't know how to propose a fancy
sophisticated machine learning algorithm for this task.

\item[Q2] Is there any restriction to use the sample input file? Can I use it
as part of my training file for NER?

\item[A2] Yes. You can do that. But keep in mind that we will use a different
input file to evaluate your pipeline, so don't overfit your model! In addition,
we encourage you to test your pipeline with the given input file to make sure
your collection reader works properly, and test your pipeline with the sample
output, and report the preliminary result in your report.

\item[Q3] How were the gene mentions (gold-standard annotations) identified in
the output file?

\item[A3] According to the \emph{GENETAG Annotation Guidelines}, there are
several rules for the annotators to follow in order to determine which words are
considered as part of a single gene/protein mention. If you are really curious
about the details, you can refer to the Appendix for the guideline.

\end{qa}

Here, we present you a very simple named entity recognizer based on each token's
part-of-speech (e.g., noun, verb, adjective, adverb, etc.) in Listing
\ref{lst:simple-ner}. The idea of this named entity recognizer is simple. It
first uses a tokenizer to tokenize the sentence into tokens, and then label each
token with a part-of-speech tag. Finally, the consecutive nouns will be grouped
into ``noun phrases'', which will be added to the list of name entities.

\small
\lstinputlisting[language=Java,float,linewidth=1.2\textwidth,caption=A simple
name entity recognition program based on Stanford CoreNLP
tool,label=lst:simple-ner]{PosTagNamedEntityRecognizer.java}
\normalsize

You can test the perform of this named entity recognizer with the collection
processing engine that you will implement later. The output for the first
sentence in the collection (i.e. the example sentence we used earlier for this
task) is\footnote{Note that the white-spaces should not be counted.}:

\begin{verbatim}
P00001606T0076|0 10|Comparison
P00001606T0076|14 33|alkaline phosphatases
P00001606T0076|37 55|5-nucleotidase
\end{verbatim}

Luckily, we correctly identify two gene mentions with this simple approach. But
``Comparison'' is definitely not a gene mention. If you like to apply a similar
approach, you could try to find out solutions to fix this. Got some ideas? I
suggest you to write them down now since you need some additional steps to setep
the project before you can actually implement the components.
