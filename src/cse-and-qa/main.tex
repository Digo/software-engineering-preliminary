% 
% This work is licensed under the Creative Commons Attribution-ShareAlike 3.0
% Unported License. To view a copy of this license, visit
% http://creativecommons.org/licenses/by-sa/3.0/.
%
% author: Zi Yang <ziy@cs.cmu.edu>
% date: 09/02/2012
%
\documentclass[oneside]{memoir}

\renewcommand{\chaptername}{Task}

\usepackage{times}

\usepackage{titlesec}
\titleformat{\section}{\normalfont\Large\bfseries}{Task \thesection}{1em}{}

\usepackage{url}
\usepackage{hyperref}

\usepackage{graphicx}
\graphicspath{{../../fig/cse-and-qa/}}

\usepackage{pstricks}
\usepackage{epsfig}

\usepackage{listings}
\usepackage{color}
\definecolor{dkgreen}{rgb}{0,0.6,0}
\definecolor{gray}{rgb}{0.5,0.5,0.5}
\definecolor{lightblue}{rgb}{0.95,0.95,1}
\definecolor{mauve}{rgb}{0.58,0,0.82}
\lstset{
  basicstyle=\ttfamily\small,
  numbers=left,
  numberstyle=\tiny\color{gray},
  stepnumber=2, 
  numbersep=5pt,
  backgroundcolor=\color{lightblue},
  showspaces=false,
  showstringspaces=false,
  showtabs=false,
  frame=lines,
  rulecolor=\color{black},
  tabsize=2,
  captionpos=b,
  breaklines=true,
  breakatwhitespace=false,
  title=\lstname,
  keywordstyle=\color{blue},
  commentstyle=\color{dkgreen},
  stringstyle=\color{mauve},
  escapeinside={\%*}{*)},
  morekeywords={*,...},
}
\usepackage{letltxmacro}
\makeatletter
\LetLtxMacro\@@lst@inputlisting\lst@inputlisting
\renewcommand\lst@inputlisting[2][]{%
  \try@listingspath{#2}%
  \if@tempswa
    \typeout{Using \@foundlisting}%
    \@@lst@inputlisting[#1]{\@foundlisting}%
  \else
    \typeout{Missing file #2}\endgroup
  \fi}
\def\listingspath#1{\def\@listingspath{#1}}
\listingspath{}
\def\try@listingspath#1{%
  \@tempswafalse
  \expandafter\@tfor\expandafter\next
  \expandafter:\expandafter=\@listingspath\do
  {\if@tempswa\@break@tfor\fi
   \IfFileExists{\next/#1}{\@tempswatrue\xdef\@foundlisting{\next/#1}}{}}%
}
\makeatletter
\listingspath{{../../lst/cse-and-qa/}}

\usepackage{multirow}

\definecolor{shadecolor}{gray}{0.9}

\newenvironment{qa}
{\begin{shaded}\begin{itemize}}
{\end{itemize}\end{shaded}}

\title{{\bfseries 11-791 Design and Engineering of Intelligent Information
System Fall 2012\\Homework 2}\\
\vspace{1em}
\itshape\rmfamily Configuration Space Exploration for Biomedical Question
Answering}

\date{}

\begin{document}

\begin{titlingpage}
\maketitle

\hspace{-0.1\textwidth}
\begin{minipage}{1.2\textwidth}
\vspace{-5em}
\textbf{Important dates}
\begin{itemize}

\item \textbf{Hand out: November 1.}\footnote{This version was built on \today}

\item \textbf{Milestone 1 (M1) turn in: November 8.}

\begin{itemize}

\item You are required to submit all your Java source codes and YAML descriptors
for your components, and you DON'T need to submit any other documents. Your main
YAML descriptor should include a single phase of multiple options for NERs.

\item In addition, please send us the URL of your project repository page (e.g.,
\url{https://github.com/ID/hw2-teamXX}). We will look into your Issues page and
Wiki page, and expect you have created milestones and issues in the Issues
system, and reported the results of your named entity recognizers evaluated on
your own, your team meeting minutes and probably your project goals or other any
important things in your Wiki page.

\end{itemize}

\item \textbf{Milestone 2 (M2) turn in: November 22.}

\begin{itemize}

\item You are required to submit all your Java source codes and YAML descriptors
for your components, and you DON'T need to submit any other documents. You main
YAML should include a complete pipeline, where each phase can only have a SINGLE
component. You can pick the best component for each phase based on your own
cross-option evaluation with CSE.

\item We will again look into your Issues page and Wiki page, and expect you
properly solved your previous created milestones and issues, and probably
created new milestones and issues, and reported the evaluation results of your
M2 on your own, your team meeting minutes and probably your revised project
goals.

\end{itemize}

\end{itemize}

\end{minipage}
\hspace{-0.1\textwidth}

\hspace{-0.1\textwidth}
\begin{minipage}{1.2\textwidth}

\begin{itemize}

\item \textbf{Milestone 3 (M3) turn in: December 6.}

\begin{itemize}

\item You are required to submit all your Java source codes and YAML descriptors
for your components. You main YAML should include a complete pipeline, where
each phase can only have a SINGLE component. You can pick the best component for
each phase based on your own cross-option evaluation with CSE.

\item Name your presentation slides as
\texttt{hw2-teamXX.ppt}\footnote{Or other formats, e.g., pptx, odp, pdf, etc.}
and put it in the \texttt{src/main/resources/docs}.

\item We expect you properly solved all the issues and accomplish all
milestones, and reported the evaluation results of your final system, your team
meeting minutes and a final summary.

\end{itemize}

\end{itemize}

\end{minipage}
\hspace{-0.1\textwidth}

\hspace{-0.1\textwidth}
\begin{minipage}{1.2\textwidth}

You should always organize your project in the same hierarchy as shown
below for all your submissions:

\small
\begin{verbatim}
hw2-teamXX
|- pom.xml
|- launches
|  `- hellobioqa.launch
|- data                           /* git-ignore this folder, since
|  `- oaqa-eval.db3                * it will become huge */
`- src
     `- main
        |- java/edu/cmu/lti/oaqa/openqa/test/teamXX
      |   `- **/*.java             /* your Java codes go into this 
      |                            * folder as you did before */
        `- resources
           |- input
           |- gs
           `- hellobioqa
              |- collection        /* descriptors for collection reader,
              |- keyterm            * example keyterm extractors,
              |- retrieval          * example retrieval strategists,
              |- passage            * example passage extractors */
              |- hellobioqa.yaml   /* the entry point for the example 
              |                     * pipeline and your pipeline */
              `- teamXX            /* your descriptors go into here */
                 `- **/*.yaml

\end{verbatim}
\normalsize

Several notes about organizing your Maven project and other additional
information:

\begin{enumerate}

\item \textbf{Submission:} The same way as you did for Homework 0 and 1 (set up
GitHub repo, create Maven project, write your code, submit to Maven repo),
except that the name has changed to hw2-teamXX (XX is a two-digit number from 01
to 18).

\item \textbf{Your source files and descriptors:}
\texttt{java/edu/cmu/lti/oaqa/openqa/test/teamXX} means you should create
directories (or packages in terms of Java program structure) iteratively from
\texttt{java}, all the way to \texttt{teamXX}.

\verb|**/*.java| and \verb|**/*.yaml| tell you that you don't need to flatten
your folder hierarchy, instead we encourage you to place your Java codes in the
right package, and similarly, you can create subfolders for different types of
descriptors, e.g., \verb|src/main/resources/hellobioqa/teamXX/keyterm| for all
the descriptors related to keyterm extraction task.

Actually, once you specify the main yaml descriptor of pipeline to the
\texttt{ECDDriver}, you can have your own entry point to the system. However, if
you want to use \texttt{launches/hellobioqa.launch} file to run the pipeline,
you should consider \texttt{src/main/resources/hellobioqa/hellobioqa.yaml} as
your main yaml. When we evaluate your codes, we will run a different main yaml
to bundle all your components, all the components declared in your
\texttt{hellobioqa.yaml} will be used.

\item \textbf{Comments, Javadocs, and documentations:} They are still important
if you want us and other users to better understand your code. (Remember: we
will become your customer when we run the big experiment.)

\end{enumerate}

\end{minipage}
\hspace{-0.1\textwidth}

\hspace{-0.1\textwidth}
\begin{minipage}{1.2\textwidth}

\textbf{Useful information}
\begin{enumerate}

\item Please visit the course blackboard regularly to check if a newer version
is published. We may have new versions or revised instruction at the begining of
each milestone. Since this is still an ongoing project, we might also fix bugs
or enhance features to the framework. If an urgent bug fix is done, we will also
make an announcement on the blackboard.

\item If you have difficulties in using Eclipse, Java, git, GitHub, Maven,
Sonatype Nexus, Lucene, Solr, etc., you will find Google is the still the most
effective way to solve any problem. But if you find the problem might exist in
the framework, the quickest solution is to post an issue on the GitHub ISSUES
system, and all the developers will be notified for the newly posted issue.

Please take a look at the following table to find out where to post an issue.

\vspace{1em}

\begin{tabular}{p{14em}lp{15em}}
\hline
Key words of your problem & Suspected project & Report it at \\
\hline
ECD, descriptor, phase, yaml, driver & uima-ecd &
\url{https://github.com/oaqa/uima-ecd/issues} \\
CSE, persistence-provider, retrieval evaluation & cse-framework &
\url{https://github.com/oaqa/cse-framework/issues} \\
QA, abstract classes, OAQA type system, keyterm evaluation, passage evaluation,
collection reader & baseqa &
\url{https://github.com/oaqa/baseqa/issues} \\
oaqa-eval, cse repository, database & jdbc-provider &
\url{https://github.com/oaqa/jdbc-provider/issues} \\
solr, retrieval, lucene & solr-provider &
\url{https://github.com/oaqa/solr-provider/issues} \\
example implementations & helloqa &
\url{https://github.com/oaqa/helloqa/issues} \\
trec-genomics adaptation & hellobioqa &
\url{https://github.com/ziy/hellobioqa/issues} \\
\hline
\end{tabular}

\vspace{1em}

For any question regarding the policy of Homework 2, please send
mails directly to Prof. Eric Nyberg
(\href{mailto:ehn@cs.cmu.edu}{\nolinkurl{ehn@cs.cmu.edu}}), and for other
questions about the homework, please send us mails: Zi Yang
(\href{mailto:ziy@cs.cmu.edu}{\nolinkurl{ziy@cs.cmu.edu}}) or Rui Liu
(\href{mailto:ruil@cs.cmu.edu}{\nolinkurl{ruil@cs.cmu.edu}}).

\item Again, both source files and pdf file of this assignment are
publicly available on my GitHub

\url{http://github.com/ziy/software-engineering-preliminary}

Please feel free to fork the project and send a pull request back to me as some
of you did for Homework 0 for any error. Or you can just report an issue at

\url{http://github.com/ziy/software-engineering-preliminary/issues}

\end{enumerate}

\end{minipage}
\hspace{-0.1\textwidth}

\end{titlingpage}


\chapter{UML Design}

In this task, you are required to answer some questions about UML. Once you've
answered all the questions, remembered to convert whatever format your document
is to PDF/DOC/DOCX, and name it as hw1-ID-report.pdf, hw1-ID-report.doc,
hw1-ID-report.docx, where ID is your Andrew ID, and put it under
\texttt{src/main/resources/docs}. Moreover, don't forget to put your name and
Andrew ID at the beginning of the document.

\section{Domain diagram for IntelligentInformationSystem}

``An IntelligentInformationSystem is composed of a sequence of data processing
operations or phases. Each phase accepts certain data types as input and
produces certain data types as output. Each phase can be implemented by any
number of algorithms or options. Each option is implemented by a specific Java
class. Each option may have any number of configuration parameters; each
configuration parameter has some set of acceptable values.''

Draw a UML Domain Diagram to represent the domain concepts, associations (with
multiplicities) and attributes expressed in the description above.

\section{Domain diagram for AnalysisEngine}

``An AnalysisEngine is composed of a sequence of algorithms or options. Each
option accepts certain data types as input and produces certain data types as
output. Each option is implemented by a specific Java class. Each option has
some number of configuration parameters; each configuration parameter has a
specific assigned value.''

Draw a UML Domain Diagram to represent the domain concepts, associations (with
multiplicities) and attributes expressed in the description above.

\section{Sequence Diagram}

There is a one-to-many relationship between IntelligentInformationSystem and
AnalysisEngine. Assume that an IntelligentInformationSystem has the
responsibility to produce a set of AnalysisEngines that represent all of the
possible data flows in the IntelligentInformationSystem. Design a method with
this signature:

\small
\begin{verbatim}
ArrayList<CollectionProcessingEngine> instantiateEngines (
                          IntelligentInformationSystem iis);
\end{verbatim}
\normalsize

Draw a UML Sequence Diagram to show the sequence of messages required to a) read
the information from the IntelligentInformationSystem instance, b) instantiate
the corresponding AnalysisEngine instances, and c) store the AnalysisEngine
instances in a List, which is the final output of the program. The message and
return value for this use case are illustrated below.

\begin{figure}[h]
\centering
\begin{pspicture}(0,-2.315)(6.1384373,2.335)
\usefont{T1}{ptm}{m}{n}
\rput(3.6746874,1.81){\psframebox[linewidth=0.04,framesep=0.3]{IntelligentInformationSystem}}
\usefont{T1}{ptm}{m}{n}
\rput(1.6684375,0.61){instantiateEngines}
\usefont{T1}{ptm}{m}{n}
\rput(1.7542187,-0.99){List<AnalysisEngine>}
\psframe[linewidth=0.04,dimen=outer](4.0,0.505)(3.6,-1.495)
\psline[linewidth=0.04cm,linestyle=dashed,dash=0.16cm 0.16cm](3.8,1.305)(3.8,0.505)
\psline[linewidth=0.04cm,linestyle=dashed,dash=0.16cm 0.16cm](3.8,-1.495)(3.8,-2.295)
\psline[linewidth=0.04cm,arrowsize=0.05291667cm 5.0,arrowlength=1.4,arrowinset=0.4]{->}(0.0,0.305)(3.6,0.305)
\psline[linewidth=0.04cm,linestyle=dashed,dash=0.16cm 0.16cm,arrowsize=0.05291667cm 5.0,arrowlength=1.4,arrowinset=0.4]{->}(3.6,-1.495)(0.0,-1.495)
\end{pspicture} 
\caption{The message and return value for this use case}
\end{figure}




\chapter{UML Design}

In this task, you are required to answer some questions about UML. Once you've
answered all the questions, remembered to convert whatever format your document
is to PDF/DOC/DOCX, and name it as hw1-ID-report.pdf, hw1-ID-report.doc,
hw1-ID-report.docx, where ID is your Andrew ID, and put it under
\texttt{src/main/resources/docs}. Moreover, don't forget to put your name and
Andrew ID at the beginning of the document.

\section{Domain diagram for IntelligentInformationSystem}

``An IntelligentInformationSystem is composed of a sequence of data processing
operations or phases. Each phase accepts certain data types as input and
produces certain data types as output. Each phase can be implemented by any
number of algorithms or options. Each option is implemented by a specific Java
class. Each option may have any number of configuration parameters; each
configuration parameter has some set of acceptable values.''

Draw a UML Domain Diagram to represent the domain concepts, associations (with
multiplicities) and attributes expressed in the description above.

\section{Domain diagram for AnalysisEngine}

``An AnalysisEngine is composed of a sequence of algorithms or options. Each
option accepts certain data types as input and produces certain data types as
output. Each option is implemented by a specific Java class. Each option has
some number of configuration parameters; each configuration parameter has a
specific assigned value.''

Draw a UML Domain Diagram to represent the domain concepts, associations (with
multiplicities) and attributes expressed in the description above.

\section{Sequence Diagram}

There is a one-to-many relationship between IntelligentInformationSystem and
AnalysisEngine. Assume that an IntelligentInformationSystem has the
responsibility to produce a set of AnalysisEngines that represent all of the
possible data flows in the IntelligentInformationSystem. Design a method with
this signature:

\small
\begin{verbatim}
ArrayList<CollectionProcessingEngine> instantiateEngines (
                          IntelligentInformationSystem iis);
\end{verbatim}
\normalsize

Draw a UML Sequence Diagram to show the sequence of messages required to a) read
the information from the IntelligentInformationSystem instance, b) instantiate
the corresponding AnalysisEngine instances, and c) store the AnalysisEngine
instances in a List, which is the final output of the program. The message and
return value for this use case are illustrated below.

\begin{figure}[h]
\centering
\begin{pspicture}(0,-2.315)(6.1384373,2.335)
\usefont{T1}{ptm}{m}{n}
\rput(3.6746874,1.81){\psframebox[linewidth=0.04,framesep=0.3]{IntelligentInformationSystem}}
\usefont{T1}{ptm}{m}{n}
\rput(1.6684375,0.61){instantiateEngines}
\usefont{T1}{ptm}{m}{n}
\rput(1.7542187,-0.99){List<AnalysisEngine>}
\psframe[linewidth=0.04,dimen=outer](4.0,0.505)(3.6,-1.495)
\psline[linewidth=0.04cm,linestyle=dashed,dash=0.16cm 0.16cm](3.8,1.305)(3.8,0.505)
\psline[linewidth=0.04cm,linestyle=dashed,dash=0.16cm 0.16cm](3.8,-1.495)(3.8,-2.295)
\psline[linewidth=0.04cm,arrowsize=0.05291667cm 5.0,arrowlength=1.4,arrowinset=0.4]{->}(0.0,0.305)(3.6,0.305)
\psline[linewidth=0.04cm,linestyle=dashed,dash=0.16cm 0.16cm,arrowsize=0.05291667cm 5.0,arrowlength=1.4,arrowinset=0.4]{->}(3.6,-1.495)(0.0,-1.495)
\end{pspicture} 
\caption{The message and return value for this use case}
\end{figure}



\end{document}
