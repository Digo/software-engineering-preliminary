
\section{Team coding with GitHub}

You will again use GitHub to host your project code. But different from previous
homeworks, all the team members need to collaborate on the project, and commit
the code changes to the same repository, which means it is good to learn how to
more about git and GitHub in this subtask.

\subsection{Creating a repository}

\begin{enumerate}

\item The team leader needs to create a repository with his/her GitHub acccount,
and name your project as \texttt{hw2-teamXX}, where \texttt{XX} is your team
number, which is a two-digit number ranging from 01 to 18. If you are not sure
how to create a GitHub repository, please refer to Homework 0.

At this point, all the team members are able to checkout the empty project and
start working on their own. But this is NOT recommended! We recommend the team
leader could checkout the project from GitHub repository, go through the entire
Maven project building process (we will come to this in the next subtask) until
you can ran a simple hellobioqa pipeline. Then, the team leader again, from
his/her laptop, commits and pushes everything to the repository before the team
members clone the project.

\item Now, everyone has the read permission to your project repository (since it
is a public repository). To help your team members gain read/write permissions,
you, the team leader, should add them as collaborators. You need to click the
\textbf{Admin} tab on your project homepage, and click the
\textbf{Collaborators} menu on the left. Put in your teammates' GitHub IDs one
by one, and click \textbf{Add} button.

\end{enumerate}

\subsection{Creating milestones, issues, and Wiki pages}

To create milestones for your development will help better organize your team
members, know your collaborators' progress, and help users/customers know what
they could expect from a future release. Issues can be detailed action items
team members would like to contribute to each milestone. Action can be bug fix,
feature enhancement, etc. If you are still unclear what a milestone is or what
an issue is, or you want to understand how milestone and issue tracking system
can help software development, you can search on Google and find out many
interesting blogs. You are recommended to read about the features of GitHub
issue tracking system at
\url{https://github.com/blog/831-issues-2-0-the-next-generation} and
\url{https://github.com/features/projects/issues}.

We recommend all the team members have a discussion on how to set the
milestones, and what your first several initial issues will be (i.e., what they
first couple of things you want to do at the beginning.) And the team leader
does the following steps. \textbf{Remember that you are unlikely to submit all
the issues to the tracking system at once, you may create new issues for bugs to
fix or new features to implement, change the owner (assignee) of the issue,
close implemented issues, or mark some as wontfix. All your team members
are responsible to maintain your milestones and issues.}

\begin{enumerate}

\item To create a milestone, you can navigate to the \textbf{Issues} tab of your
project homepage, then click the \textbf{Milestones} below the main menu bar.
Now you are able to see the button \textbf{Create a new milestone}, click it.

\item Type your title for the milestone, e.g., ``M1'' for the first milestone
(you can also make it more meaningful and specific). Write a few descriptions
for this milestone, like the goal and the brief descriptions of features will be
enhanced in this milestone. Finally, select a ``Due Data'' for the milestone.
Click the ``Create Milestone'' to finalize creating your milestone.

\item Do the previous step once again until all your milestones are created.

\item Now, you need to submit your first several issues. Go back to the
\textbf{Issues} tab on your project homepage, and click \textbf{New Issue}. Type
a title, assign the task to a particular team member, link this issue to a
previously created milestone. Finally write down detailed comments to the issue.

You may find when you type ``@'' followed by your collaborator's ID in the
comment box, you are able to mention your collaborator as you mention your
friend in a tweet on Twitter.

You can learn how to write in GitHub Flavored Markdown language, by clicking the
link above the textbox.

You can also attch labels to each issue by selecting them from the right panel.
As we mentioned earlier, you can change the milestone assignment, in particular,
if you find you couldn't finish it by M1, then you can change it to M2 later, or
you can also relabel it as \textbf{wontfix}.

\end{enumerate}

Now, you can create a Wiki page for your team meeting minutes and other
important items.

\begin{enumerate}

\item Click the ``Wiki'' tab at the top of your project homepage, and click
\textbf{Edit Page} to start editing it.

\end{enumerate}

Once you reach this point, remember to send us an email to report the URL of
your project repository page (e.g., \url{https://github.com/ID/hw2-teamXX}).

\subsection{Team coding with git-branch, git-merge, git-rebase}

Collaboration is important for this homework. All the team members start their
individual development after the team leader gets the framework ready, and
pushes all his/her local commits to the repository. Once a team member finalizes
his/her development, he/she might take responsibility on another task, or want
to check the integrity or compatibility with features implemented by
collaborators. After all the individual developments are done, team leader is
responsible to merge all the newly developed components into the same codebase
and test the integrity.

Git branching is a good tool to help the team manage parallel development and
distributed codebase. In fact, the branching mechanism is widely adopted by not
only git, but many other version control systems, e.g., SVN or Mercurial (Hg).
But one of the most important reasons that people love git branching over SVN or
Mercurial (Hg) is its light-weight nature, which allow switching between
development branches within the same clone of a repository.

Normally, the team leader is in charge of the \texttt{master} branch (the one
you probably used for homework 0 and 1), which usually corresponds to a codebase
for the most recent stable release, while team members should create branches
for each individual task assignment, e.g., bug fixes, feature enhancement with
\texttt{git branch} command. You can also do it within Eclipse by right-clicking
the project name, and select \textbf{Team} $\rightarrow$ \textbf{Switch To}
$\rightarrow$ \textbf{New Branch\ldots}.

To merge multiple development branches back to \texttt{master} branch (or in
Eclipse \textbf{Team} $\rightarrow$ \textbf{Switch To} $\rightarrow$
\textbf{master}), you should switch back to master, and then \texttt{git merge}
the branch you want to be merged (or \textbf{Team} $\rightarrow$
\textbf{Merge\ldots}). Sometimes, you may also need \texttt{git rebase} when you
later realize your development should depend on another feature that was also
being developed by your collaborator, which hadn't beed integrated in the
\texttt{master} branch by the time you created your development branch.

To better understand git branching, you should read the ``Git Branching''
chapter of Pro Git Book (\url{http://git-scm.com/book/en/Git-Branching}). You
can also find a more sophisticated branching model at
\url{http://nvie.com/posts/a-successful-git-branching-model/}, which may be too
complicated for this homework, but it can inspire how you want to manage your
branches.

If you are looking for a Eclipse plug-in that can bring the GitHub issue
tracking system to your workspace, e.g., create/close/comment issues, label
them, assign to a person within eclipse, or automatically get notified by a
message bubble if an issue is assigned to you, then you can try to investigate
Mylyn plug-in. In the ``Tips and Tricks using Eclipse with Github''
post\footnote{
\url{http://eclipsesource.com/blogs/2012/08/28/tips-and-tricks-using-eclipse-with-github/}}
, you will find the basic idea how Mylyn GitHub connector works. By default,
Eclipse Juno for Java developers comes with EGit, Mylyn, and Mylyn Github
connector, and the Task View is on the top-right corner of the Java perspective
by default.
