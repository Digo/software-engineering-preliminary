
\section{Planning on keyterm extractor based on abstract classes}

Different from what you did for Homework 1, where you built the whole UIMA CPE
pipeline all by yourself, by composing \texttt{Annotator}s and writing XML
descriptors, UIMA ECD \& CSE framework provide you with easy-to-write YAML based
descriptors and easy-to-use execution module. Moreover, \texttt{BaseQA} project
includes abstract classes that extend from \verb|JCasAnnotator_ImplBase|. For
this task, you need to focus on two abstract classes defined for the first
phrase \emph{keyterm extraction}: \texttt{AbstractKeytermExtractor} and
\texttt{AbstractKeytermUpdater}.

Based on your experience with UIMA SDK, there are two subtasks to migrate your
code from UIMA SDK to CSE: Java code migration and descriptor migration. We will
help you with both of them in the next sections. But before you move onto the
next section, we encourage the teams should clearly plan and identify the tasks
(by creating issues on GitHub and assigning each to a particular team member)
before start coding. For example, if you have four members in the team, then it
might be a good idea to create four different issues indicating four different
integration subtasks.

\begin{qa}

\item[Q1] What if all of us implemented exactly the same algorithm for Homework
1? Do we still need to incorporate all of them?

\item[A1] Yes. For M1, it is just a practise for all the team members, and you
should report the comparison experiment even though some of them might show the
same performance in evaluation results. If you want to see a different result,
we encourage you to include the baseline NER method we provided to you in
hw1-archetype, and integrate it as another option.

For M2 and M3, you only need to incorporate a single best component for each
phrase.

\item[Q2] I notice that the gold-standard keyterms are not only gene or protein
names, and it also makes sense to me that keyterms should contain other types of
important information, like key verbs. But our implementation for Homework 1
focused on gene and protein name identification, should I make any change to my
code to in order to identify other types of keyterms?

\item[A2] Generally speaking, you don't need for M1. We won't grade your M1
based on keyterm extraction results. But for M2 and M3, you may find sometimes
gene and protein names are not enough to express the information need.

\end{qa}
